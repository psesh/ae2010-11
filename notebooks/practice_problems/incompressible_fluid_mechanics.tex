% Options for packages loaded elsewhere
\PassOptionsToPackage{unicode}{hyperref}
\PassOptionsToPackage{hyphens}{url}
\PassOptionsToPackage{dvipsnames,svgnames,x11names}{xcolor}
%
\documentclass[
  1.2em,
  letterpaper,
  DIV=11,
  numbers=noendperiod]{scrartcl}

\usepackage{amsmath,amssymb}
\usepackage{setspace}
\usepackage{iftex}
\ifPDFTeX
  \usepackage[T1]{fontenc}
  \usepackage[utf8]{inputenc}
  \usepackage{textcomp} % provide euro and other symbols
\else % if luatex or xetex
  \usepackage{unicode-math}
  \defaultfontfeatures{Scale=MatchLowercase}
  \defaultfontfeatures[\rmfamily]{Ligatures=TeX,Scale=1}
\fi
\usepackage{lmodern}
\ifPDFTeX\else  
    % xetex/luatex font selection
\fi
% Use upquote if available, for straight quotes in verbatim environments
\IfFileExists{upquote.sty}{\usepackage{upquote}}{}
\IfFileExists{microtype.sty}{% use microtype if available
  \usepackage[]{microtype}
  \UseMicrotypeSet[protrusion]{basicmath} % disable protrusion for tt fonts
}{}
\makeatletter
\@ifundefined{KOMAClassName}{% if non-KOMA class
  \IfFileExists{parskip.sty}{%
    \usepackage{parskip}
  }{% else
    \setlength{\parindent}{0pt}
    \setlength{\parskip}{6pt plus 2pt minus 1pt}}
}{% if KOMA class
  \KOMAoptions{parskip=half}}
\makeatother
\usepackage{xcolor}
\setlength{\emergencystretch}{3em} % prevent overfull lines
\setcounter{secnumdepth}{-\maxdimen} % remove section numbering
% Make \paragraph and \subparagraph free-standing
\ifx\paragraph\undefined\else
  \let\oldparagraph\paragraph
  \renewcommand{\paragraph}[1]{\oldparagraph{#1}\mbox{}}
\fi
\ifx\subparagraph\undefined\else
  \let\oldsubparagraph\subparagraph
  \renewcommand{\subparagraph}[1]{\oldsubparagraph{#1}\mbox{}}
\fi

\usepackage{color}
\usepackage{fancyvrb}
\newcommand{\VerbBar}{|}
\newcommand{\VERB}{\Verb[commandchars=\\\{\}]}
\DefineVerbatimEnvironment{Highlighting}{Verbatim}{commandchars=\\\{\}}
% Add ',fontsize=\small' for more characters per line
\usepackage{framed}
\definecolor{shadecolor}{RGB}{241,243,245}
\newenvironment{Shaded}{\begin{snugshade}}{\end{snugshade}}
\newcommand{\AlertTok}[1]{\textcolor[rgb]{0.68,0.00,0.00}{#1}}
\newcommand{\AnnotationTok}[1]{\textcolor[rgb]{0.37,0.37,0.37}{#1}}
\newcommand{\AttributeTok}[1]{\textcolor[rgb]{0.40,0.45,0.13}{#1}}
\newcommand{\BaseNTok}[1]{\textcolor[rgb]{0.68,0.00,0.00}{#1}}
\newcommand{\BuiltInTok}[1]{\textcolor[rgb]{0.00,0.23,0.31}{#1}}
\newcommand{\CharTok}[1]{\textcolor[rgb]{0.13,0.47,0.30}{#1}}
\newcommand{\CommentTok}[1]{\textcolor[rgb]{0.37,0.37,0.37}{#1}}
\newcommand{\CommentVarTok}[1]{\textcolor[rgb]{0.37,0.37,0.37}{\textit{#1}}}
\newcommand{\ConstantTok}[1]{\textcolor[rgb]{0.56,0.35,0.01}{#1}}
\newcommand{\ControlFlowTok}[1]{\textcolor[rgb]{0.00,0.23,0.31}{#1}}
\newcommand{\DataTypeTok}[1]{\textcolor[rgb]{0.68,0.00,0.00}{#1}}
\newcommand{\DecValTok}[1]{\textcolor[rgb]{0.68,0.00,0.00}{#1}}
\newcommand{\DocumentationTok}[1]{\textcolor[rgb]{0.37,0.37,0.37}{\textit{#1}}}
\newcommand{\ErrorTok}[1]{\textcolor[rgb]{0.68,0.00,0.00}{#1}}
\newcommand{\ExtensionTok}[1]{\textcolor[rgb]{0.00,0.23,0.31}{#1}}
\newcommand{\FloatTok}[1]{\textcolor[rgb]{0.68,0.00,0.00}{#1}}
\newcommand{\FunctionTok}[1]{\textcolor[rgb]{0.28,0.35,0.67}{#1}}
\newcommand{\ImportTok}[1]{\textcolor[rgb]{0.00,0.46,0.62}{#1}}
\newcommand{\InformationTok}[1]{\textcolor[rgb]{0.37,0.37,0.37}{#1}}
\newcommand{\KeywordTok}[1]{\textcolor[rgb]{0.00,0.23,0.31}{#1}}
\newcommand{\NormalTok}[1]{\textcolor[rgb]{0.00,0.23,0.31}{#1}}
\newcommand{\OperatorTok}[1]{\textcolor[rgb]{0.37,0.37,0.37}{#1}}
\newcommand{\OtherTok}[1]{\textcolor[rgb]{0.00,0.23,0.31}{#1}}
\newcommand{\PreprocessorTok}[1]{\textcolor[rgb]{0.68,0.00,0.00}{#1}}
\newcommand{\RegionMarkerTok}[1]{\textcolor[rgb]{0.00,0.23,0.31}{#1}}
\newcommand{\SpecialCharTok}[1]{\textcolor[rgb]{0.37,0.37,0.37}{#1}}
\newcommand{\SpecialStringTok}[1]{\textcolor[rgb]{0.13,0.47,0.30}{#1}}
\newcommand{\StringTok}[1]{\textcolor[rgb]{0.13,0.47,0.30}{#1}}
\newcommand{\VariableTok}[1]{\textcolor[rgb]{0.07,0.07,0.07}{#1}}
\newcommand{\VerbatimStringTok}[1]{\textcolor[rgb]{0.13,0.47,0.30}{#1}}
\newcommand{\WarningTok}[1]{\textcolor[rgb]{0.37,0.37,0.37}{\textit{#1}}}

\providecommand{\tightlist}{%
  \setlength{\itemsep}{0pt}\setlength{\parskip}{0pt}}\usepackage{longtable,booktabs,array}
\usepackage{calc} % for calculating minipage widths
% Correct order of tables after \paragraph or \subparagraph
\usepackage{etoolbox}
\makeatletter
\patchcmd\longtable{\par}{\if@noskipsec\mbox{}\fi\par}{}{}
\makeatother
% Allow footnotes in longtable head/foot
\IfFileExists{footnotehyper.sty}{\usepackage{footnotehyper}}{\usepackage{footnote}}
\makesavenoteenv{longtable}
\usepackage{graphicx}
\makeatletter
\def\maxwidth{\ifdim\Gin@nat@width>\linewidth\linewidth\else\Gin@nat@width\fi}
\def\maxheight{\ifdim\Gin@nat@height>\textheight\textheight\else\Gin@nat@height\fi}
\makeatother
% Scale images if necessary, so that they will not overflow the page
% margins by default, and it is still possible to overwrite the defaults
% using explicit options in \includegraphics[width, height, ...]{}
\setkeys{Gin}{width=\maxwidth,height=\maxheight,keepaspectratio}
% Set default figure placement to htbp
\makeatletter
\def\fps@figure{htbp}
\makeatother

\KOMAoption{captions}{tableheading}
\makeatletter
\makeatother
\makeatletter
\makeatother
\makeatletter
\@ifpackageloaded{caption}{}{\usepackage{caption}}
\AtBeginDocument{%
\ifdefined\contentsname
  \renewcommand*\contentsname{Table of contents}
\else
  \newcommand\contentsname{Table of contents}
\fi
\ifdefined\listfigurename
  \renewcommand*\listfigurename{List of Figures}
\else
  \newcommand\listfigurename{List of Figures}
\fi
\ifdefined\listtablename
  \renewcommand*\listtablename{List of Tables}
\else
  \newcommand\listtablename{List of Tables}
\fi
\ifdefined\figurename
  \renewcommand*\figurename{Figure}
\else
  \newcommand\figurename{Figure}
\fi
\ifdefined\tablename
  \renewcommand*\tablename{Table}
\else
  \newcommand\tablename{Table}
\fi
}
\@ifpackageloaded{float}{}{\usepackage{float}}
\floatstyle{ruled}
\@ifundefined{c@chapter}{\newfloat{codelisting}{h}{lop}}{\newfloat{codelisting}{h}{lop}[chapter]}
\floatname{codelisting}{Listing}
\newcommand*\listoflistings{\listof{codelisting}{List of Listings}}
\makeatother
\makeatletter
\@ifpackageloaded{caption}{}{\usepackage{caption}}
\@ifpackageloaded{subcaption}{}{\usepackage{subcaption}}
\makeatother
\makeatletter
\@ifpackageloaded{tcolorbox}{}{\usepackage[skins,breakable]{tcolorbox}}
\makeatother
\makeatletter
\@ifundefined{shadecolor}{\definecolor{shadecolor}{rgb}{.97, .97, .97}}
\makeatother
\makeatletter
\makeatother
\makeatletter
\makeatother
\ifLuaTeX
  \usepackage{selnolig}  % disable illegal ligatures
\fi
\IfFileExists{bookmark.sty}{\usepackage{bookmark}}{\usepackage{hyperref}}
\IfFileExists{xurl.sty}{\usepackage{xurl}}{} % add URL line breaks if available
\urlstyle{same} % disable monospaced font for URLs
\hypersetup{
  pdftitle={Incompressible fluid mechanics},
  colorlinks=true,
  linkcolor={blue},
  filecolor={Maroon},
  citecolor={Blue},
  urlcolor={Blue},
  pdfcreator={LaTeX via pandoc}}

\title{Incompressible fluid mechanics}
\author{}
\date{}

\begin{document}
\maketitle
\ifdefined\Shaded\renewenvironment{Shaded}{\begin{tcolorbox}[boxrule=0pt, borderline west={3pt}{0pt}{shadecolor}, interior hidden, sharp corners, breakable, enhanced, frame hidden]}{\end{tcolorbox}}\fi

\renewcommand*\contentsname{Table of contents}
{
\hypersetup{linkcolor=}
\setcounter{tocdepth}{3}
\tableofcontents
}
\setstretch{1.5}
\hypertarget{problem-1}{%
\subsection{Problem 1}\label{problem-1}}

A mixing tank has two inlets and one outlet, all of the same area,
\(0.01 \; m^2\). Liquids of density
\(\require{color}{\color[rgb]{0.918231,0.469102,0.038229}\rho_1 } = {\color[rgb]{0.918231,0.469102,0.038229}800 \; kg/m^3}\)
and
\(\require{color}{\color[rgb]{0.918231,0.469102,0.038229}\rho_2 } = {\color[rgb]{0.918231,0.469102,0.038229}900 \; kg/m^3}\)
flow into the tank through separate inlets. They have uniform inlet
velocities of
\(\require{color}{\color[rgb]{0.059472,0.501943,0.998465}3.1 \; m/s}\)
and
\(\require{color}{\color[rgb]{0.059472,0.501943,0.998465}1.9 \; m/s}\)
respectively. Mixing is assumed to be complete, and the conditions at
the tank outlet are uniform. Assuming steady state conditions, find

\begin{itemize}
\item
  the mass flows into and out of the tank
\item
  the mean density and velocity of the mixture flowing out of the tank
\end{itemize}

Note that the mean velocity can be calculated based on the assumption of
incompressible flow.

Solution

Consider the schematic shown below.

\[
\large
\require{color}\dot{m}_{1} = {\color[rgb]{0.918231,0.469102,0.038229}\rho_1 } A {\color[rgb]{0.059472,0.501943,0.998465}v_1}, \; \; \; \require{color}\dot{m}_{2} = {\color[rgb]{0.918231,0.469102,0.038229}\rho_2 } A {\color[rgb]{0.059472,0.501943,0.998465}v_2}
\]

As conservation of mass must hold within the control volume, we have

\[
\large
\require{color}\dot{m}_{1} + \dot{m}_{2} = \dot{m}_{3}
\]

Note that since both liquids flowing into the tank are incompressible,
we can use volumetric flow rates, i.e.,

\[
\large
\require{color}A {\color[rgb]{0.059472,0.501943,0.998465}v_1} + A {\color[rgb]{0.059472,0.501943,0.998465}v_2} = A {\color[rgb]{0.059472,0.501943,0.998465}v_3}
\]

Noting that the areas are the same,

\[
\large
\require{color}{\color[rgb]{0.059472,0.501943,0.998465}v_1} + {\color[rgb]{0.059472,0.501943,0.998465}v_2} = {\color[rgb]{0.059472,0.501943,0.998465}v_3}
\]

Assigning values to the equations above, yields:

\begin{Shaded}
\begin{Highlighting}[]
\NormalTok{v\_1 }\OperatorTok{=} \FloatTok{3.1}
\NormalTok{v\_2 }\OperatorTok{=} \FloatTok{1.9}
\NormalTok{rho\_1 }\OperatorTok{=} \DecValTok{800}
\NormalTok{rho\_2 }\OperatorTok{=} \DecValTok{900}
\NormalTok{A }\OperatorTok{=} \FloatTok{0.01}

\CommentTok{\# From the equations above, we have the massflows in and out of the tank to be:}
\NormalTok{m\_1 }\OperatorTok{=}\NormalTok{ rho\_1 }\OperatorTok{*}\NormalTok{ v\_1 }\OperatorTok{*}\NormalTok{ A}
\NormalTok{m\_2 }\OperatorTok{=}\NormalTok{ rho\_2 }\OperatorTok{*}\NormalTok{ v\_2 }\OperatorTok{*}\NormalTok{ A}
\NormalTok{m\_3 }\OperatorTok{=}\NormalTok{ m\_1 }\OperatorTok{+}\NormalTok{ m\_2}

\BuiltInTok{print}\NormalTok{(}\StringTok{\textquotesingle{}The massflows are: m\_1 \textquotesingle{}}\OperatorTok{+}\BuiltInTok{str}\NormalTok{(m\_1)}\OperatorTok{+}\StringTok{\textquotesingle{} kg/s, m\_2 \textquotesingle{}}\OperatorTok{+}\BuiltInTok{str}\NormalTok{(m\_2)}\OperatorTok{+}\StringTok{\textquotesingle{} kg/s, and m\_3 \textquotesingle{}}\OperatorTok{+}\BuiltInTok{str}\NormalTok{(m\_3)}\OperatorTok{+}\StringTok{\textquotesingle{} kg/s\textquotesingle{}}\NormalTok{)}
\NormalTok{v\_3 }\OperatorTok{=}\NormalTok{ v\_1 }\OperatorTok{+}\NormalTok{ v\_2}
\NormalTok{rho\_3 }\OperatorTok{=}\NormalTok{ m\_3 }\OperatorTok{/}\NormalTok{ (v\_3 }\OperatorTok{*}\NormalTok{ A)}

\BuiltInTok{print}\NormalTok{(}\StringTok{\textquotesingle{}The mass density of the outflow is \textquotesingle{}}\OperatorTok{+}\BuiltInTok{str}\NormalTok{(rho\_3)}\OperatorTok{+}\StringTok{\textquotesingle{} kg/m\^{}3, with a velocity of \textquotesingle{}}\OperatorTok{+}\BuiltInTok{str}\NormalTok{(v\_3)}\OperatorTok{+}\StringTok{\textquotesingle{} m/s\textquotesingle{}}\NormalTok{)}
\end{Highlighting}
\end{Shaded}

\begin{verbatim}
The massflows are: m_1 24.8 kg/s, m_2 17.1 kg/s, and m_3 41.900000000000006 kg/s
The mass density of the outflow is 838.0000000000001 kg/m^3, with a velocity of 5.0 m/s
\end{verbatim}

\hypertarget{problem-2}{%
\subsection{Problem 2}\label{problem-2}}

A cylindrical tank with base area \(A\) and maximum height \(H\)
receives a steady flow rate \(\dot{m}_1\) from a tank, and drains a mass
flow rate \(\dot{m}_2\). The inlet pipe has a radius \(R\) and uniform
velocity
\(\require{color}{\color[rgb]{0.059472,0.501943,0.998465}V_0 \; m/s}\).
The outlet pipe also has the same radius \(R\) and a velocity profile

\[
\large
\require{color}{\color[rgb]{0.059472,0.501943,0.998465}v}(r)={\color[rgb]{0.059472,0.501943,0.998465}V_0 } \left( 1- \left( \frac{r}{R} \right)^2 \right)
\]

At time \(t = 0\), the volume of liquid in the tank is at level \(h_o\)
from the base. The density of the fluid is constant and equal to
\(\require{color}{\color[rgb]{0.918231,0.469102,0.038229}\rho}\). In
each case, determine the expression as a function of
\(R, {\color[rgb]{0.059472,0.501943,0.998465}V_0 }, A, H, ho,\) and
\(\require{color}{\color[rgb]{0.918231,0.469102,0.038229}\rho}\).

\begin{itemize}
\item
  Determine the mass flow rates through the inlet and outlet.
\item
  Determine the rate of mass accumulation in the tank.
\end{itemize}

Solution

To begin, consider the schematic above. Recognize that the massflow rate
going into the tank is given by

\[
\large
\require{color}\dot{m}_1 = {\color[rgb]{0.918231,0.469102,0.038229}\rho}A_1 {\color[rgb]{0.059472,0.501943,0.998465}v_1} = {\color[rgb]{0.918231,0.469102,0.038229}\rho} \pi R^2 {\color[rgb]{0.059472,0.501943,0.998465}V_0 }.
\]

The massflow rate leaving the tank is given by:

\[
\large
\require{color}\dot{m}_2 = \int {\color[rgb]{0.918231,0.469102,0.038229}\rho} {\color[rgb]{0.059472,0.501943,0.998465}v} dA = \int_{0}^{R} {\color[rgb]{0.918231,0.469102,0.038229}\rho} {\color[rgb]{0.059472,0.501943,0.998465}v}(r) 2 \pi r dr
\]

\[
\large
\require{color}\dot{m}_2 = \int_{0}^{R} {\color[rgb]{0.918231,0.469102,0.038229}\rho} {\color[rgb]{0.059472,0.501943,0.998465}V_0 } \left( 1- \left( \frac{r}{R} \right)^2 \right) 2 \pi r dr
\]

\[
\large
\require{color}\dot{m}_2 =  {\color[rgb]{0.918231,0.469102,0.038229}\rho} {\color[rgb]{0.059472,0.501943,0.998465}V_0 } 2 \pi \left[ \frac{r^2}{2} - \frac{r^4}{4R^2} \right]^{R}_{0} = {\color[rgb]{0.918231,0.469102,0.038229}\rho} {\color[rgb]{0.059472,0.501943,0.998465}V_0 } \pi \frac{R^2}{2}
\]

The mass accumulation in the tank is given by

\[
\large 
\require{color}\frac{dm}{dt} = \dot{m}_1 - \dot{m}_2  = {\color[rgb]{0.918231,0.469102,0.038229}\rho} \pi R^2 {\color[rgb]{0.059472,0.501943,0.998465}V_0 } - {\color[rgb]{0.918231,0.469102,0.038229}\rho} {\color[rgb]{0.059472,0.501943,0.998465}V_0 } \pi \frac{R^2}{2} = {\color[rgb]{0.918231,0.469102,0.038229}\rho} {\color[rgb]{0.059472,0.501943,0.998465}V_0 } \pi \frac{R^2}{2}
\]

\hypertarget{problem-3}{%
\subsection{Problem 3}\label{problem-3}}

A submerged submarine is towed horizontally at a steady speed
\(\require{color}{\color[rgb]{0.059472,0.501943,0.998465}U}\) in deep,
still water. Far behind at a fixed distance from the submarine an
axially-symmetrical wake is formed in which the water velocity may be
assumed to vary from
\(\require{color}{\color[rgb]{0.059472,0.501943,0.998465}U}\) on the
axis to zero at radius R (see schematic) as follows:

\[
\large
\require{color}{\color[rgb]{0.059472,0.501943,0.998465}u}(r)={\color[rgb]{0.059472,0.501943,0.998465}U } \left( 1- \left( \frac{r}{R} \right)^2 \right)
\]

The variation of the water pressure with depth may be assumed to be
unaffected by the presence of the submarine. The density of the water is
\(\require{color}{\color[rgb]{0.918231,0.469102,0.038229}\rho}\).

\begin{itemize}
\item
  To use the continuity and steady flow momentum equations we employ a
  moving control volume, in which the submarine appears stationary. Why
  do we need to do this? What is the velocity distribution in the wake
  relative to the moving control volume?
\item
  Show that the drag on the submarine is given by:
\end{itemize}

\[
\large
\require{color}{\color[rgb]{0.986252,0.007236,0.027423}F_{d}} = \frac{1}{6} \pi {\color[rgb]{0.918231,0.469102,0.038229}\rho }{\color[rgb]{0.059472,0.501943,0.998465}U}^2 R^2 
\]

\emph{Hint: Before applying momentum equation, make sure all mass flows
are accounted for in the control volume used. There are two obvious
control volumes to take, which should give the same answer.}

\begin{longtable}[]{@{}
  >{\centering\arraybackslash}p{(\columnwidth - 2\tabcolsep) * \real{0.5000}}
  >{\centering\arraybackslash}p{(\columnwidth - 2\tabcolsep) * \real{0.5000}}@{}}
\toprule\noalign{}
\begin{minipage}[b]{\linewidth}\centering
Submarine
\end{minipage} & \begin{minipage}[b]{\linewidth}\centering
Schematc
\end{minipage} \\
\midrule\noalign{}
\endhead
\bottomrule\noalign{}
\endlastfoot
& \\
\end{longtable}

Solution

A moving control volume is required to make this a steady flow problem.
Consider a control volume moving with speed
\(\require{color}{\color[rgb]{0.059472,0.501943,0.998465}u}(r)={\color[rgb]{0.059472,0.501943,0.998465}U}\).

Here the velocity is given by

\[
\large
\require{color}{\color[rgb]{0.059472,0.501943,0.998465}u}(r)={\color[rgb]{0.059472,0.501943,0.998465}U } \left( \frac{r}{R} \right)^2 
\]

where \(r\) goes from \(0\) to \(R\). We can define a cylindrical volume
that is created by the submarine, where the velocity profile essentially
mirrors that of the expression above.

Now as we are only interested in the momentum along the horizontal
direction, we can neglect the hydrostatic pressure contributions as they
are the same at both ends.

The first task is to apply continuity to quantify the massflow out of
the side of the control volume. This is because the inflow at the front
\emph{is not equal to} the outflow at the rear. The fluid that leaves
the sides does have momentum along the horizontal direction. At the
inlet, we have

\[
\large
\require{color}\dot{m}_{in} = {\color[rgb]{0.918231,0.469102,0.038229}\rho } \pi R^2 {\color[rgb]{0.059472,0.501943,0.998465}U }
\]

and at the outflow at the rear:

\[
\large
\require{color}\dot{m}_{out} = \int_{0}^{R} {\color[rgb]{0.918231,0.469102,0.038229}\rho } \pi {\color[rgb]{0.059472,0.501943,0.998465}U } 2 \pi r dr = \frac{1}{2} {\color[rgb]{0.918231,0.469102,0.038229}\rho } \pi R^2 {\color[rgb]{0.059472,0.501943,0.998465}U }
\]

From conservation of mass, we know that the massflow that exits at the
sides must be

\[
\large
\require{color}\dot{m}_{sides} = \dot{m}_{out} - \dot{m}_{in} = \frac{1}{2} {\color[rgb]{0.918231,0.469102,0.038229}\rho } \pi R^2 {\color[rgb]{0.059472,0.501943,0.998465}U }.
\]

To work out the drag, we setup the steady flow momentum equation along
the horizontal direction, recognizing that the drag force is pointing
along the positive horizontal direction (given that in our control
volume the velocity is pointing along the negative horizontal
direction).

\[
\large
\require{color}F_{{\color[rgb]{0.986252,0.007236,0.027423}d}} = \sum \dot{m}_{out} {\color[rgb]{0.059472,0.501943,0.998465}v_{out}} - \sum \dot{m}_{in} {\color[rgb]{0.059472,0.501943,0.998465}v_{in}}
\]

\[
\large
\require{color}F_{{\color[rgb]{0.986252,0.007236,0.027423}d}} = \underbrace{-\int {\color[rgb]{0.918231,0.469102,0.038229}\rho} {\color[rgb]{0.059472,0.501943,0.998465}u}^2 dA }_{\textsf{outflow at back less than zero (neg. vel)}} + \underbrace{\dot{m}_{side}\left(-{\color[rgb]{0.059472,0.501943,0.998465}U} \right)}_{\textsf{outflow on sides}}  - \underbrace{\dot{m}_{in} \left(-{\color[rgb]{0.059472,0.501943,0.998465}U} \right)}_{\textsf{inflow at front}}
\]

\[
\large
\require{color}F_{{\color[rgb]{0.986252,0.007236,0.027423}d}} = -\int_{0}^{R} {\color[rgb]{0.918231,0.469102,0.038229}\rho } \left( {\color[rgb]{0.059472,0.501943,0.998465}U } \left( \frac{r}{R} \right)^2  \right)^2 2 \pi r dr + \frac{1}{2} {\color[rgb]{0.918231,0.469102,0.038229}\rho } \pi R^2 {\color[rgb]{0.059472,0.501943,0.998465}U} \left(-{\color[rgb]{0.059472,0.501943,0.998465}U}\right) + {\color[rgb]{0.918231,0.469102,0.038229}\rho } \pi R^2 {\color[rgb]{0.059472,0.501943,0.998465}U}^2 
\]

\[
\large
\require{color}F_{{\color[rgb]{0.986252,0.007236,0.027423}d}} = \frac{1}{2} {\color[rgb]{0.918231,0.469102,0.038229}\rho } \pi R^2 {\color[rgb]{0.059472,0.501943,0.998465}U}^2 - \frac{2 \pi {\color[rgb]{0.918231,0.469102,0.038229}\rho  {\color[rgb]{0.059472,0.501943,0.998465}U}^2} }{R^4} \left[ \frac{R^6}{6} \right]
\]

\[
\large
\require{color}F_{{\color[rgb]{0.986252,0.007236,0.027423}d}} = \frac{1}{6}\pi {\color[rgb]{0.918231,0.469102,0.038229}\rho  {\color[rgb]{0.059472,0.501943,0.998465}U}}^2 R^2
\]

\hypertarget{problem-4}{%
\subsection{Problem 4}\label{problem-4}}

Two parallel streams of an incompressible fluid flowing in horizontal
rectangular ducts of height \(h\) and depth \(d\) come together at the
location \(AA'\) as shown in the figure. They have the same static
pressure
\(\require{color}{\color[rgb]{0.315209,0.728565,0.037706}p_{A}}\) and
speeds \(V\) and \(3V\) respectively. The two streams mix over a short
distance due to the turbulence generated by the unstable shear layer
between them.

Assuming the viscous shear stress on the solid surfaces can be
neglected, calculate the velocity and static pressure at location
\(BB'\), where the mixing is complete.

Solution

To work out the velocity, we use conservation of mass, i.e.,

\[
\large
\require{color} \require{color}{\color[rgb]{0.918231,0.469102,0.038229}\rho} {\color[rgb]{0.059472,0.501943,0.998465}V } dh + \require{color}{\color[rgb]{0.918231,0.469102,0.038229}\rho}{\color[rgb]{0.059472,0.501943,0.998465}3V } dh = \require{color}{\color[rgb]{0.918231,0.469102,0.038229}\rho} {\color[rgb]{0.059472,0.501943,0.998465}V'} d 2h
\]

\[
\large
\require{color} \Rightarrow {\color[rgb]{0.059472,0.501943,0.998465}V'} = {\color[rgb]{0.059472,0.501943,0.998465}2V}
\]

where \(\require{color}{\color[rgb]{0.059472,0.501943,0.998465}V'}\) is
the exit velocity.

To work out the static pressure, we use the conservation of momentum,
i.e., the force on the control volume is the net flow of momentum out.

\[
\large
\require{color}{\color[rgb]{0.315209,0.728565,0.037706}p_{A}} 2hd - {\color[rgb]{0.315209,0.728565,0.037706}p_{B}}2hd = {\color[rgb]{0.918231,0.469102,0.038229}\rho}{\color[rgb]{0.059472,0.501943,0.998465}V'} 2hd{\color[rgb]{0.059472,0.501943,0.998465}V'} - {\color[rgb]{0.918231,0.469102,0.038229}\rho} {\color[rgb]{0.059472,0.501943,0.998465}V} hd{\color[rgb]{0.059472,0.501943,0.998465}V} - {\color[rgb]{0.918231,0.469102,0.038229}\rho} 3{\color[rgb]{0.059472,0.501943,0.998465}V} hd3{\color[rgb]{0.059472,0.501943,0.998465}V}
\]

\[
\large
\require{color}
{\color[rgb]{0.315209,0.728565,0.037706}p_{A}} - {\color[rgb]{0.315209,0.728565,0.037706}p_{B}} = \frac{{\color[rgb]{0.918231,0.469102,0.038229}\rho} h d}{2 h d}\left[ 2 \left(2 {\color[rgb]{0.059472,0.501943,0.998465}V}\right)^2 - {\color[rgb]{0.059472,0.501943,0.998465}V}^2 - 9{\color[rgb]{0.059472,0.501943,0.998465}V}^2\right] = - \frac{2 {\color[rgb]{0.918231,0.469102,0.038229}\rho} h d{\color[rgb]{0.059472,0.501943,0.998465}V}^2}{2hd}
\]

\[
\large
\require{color}
{\color[rgb]{0.315209,0.728565,0.037706}p_{B}} = {\color[rgb]{0.315209,0.728565,0.037706}p_{A}} + {\color[rgb]{0.918231,0.469102,0.038229}\rho} {\color[rgb]{0.059472,0.501943,0.998465}V}^2
\]

\{\color[rgb]{0.315209,0.728565,0.037706}p\}

\hypertarget{problem-5}{%
\subsection{Problem 5}\label{problem-5}}

Consider a nozzle as shown in the diagram in below. The nozzle converges
gradually, and we assume that the flow in it is:

\begin{itemize}
\tightlist
\item
  approximately uniform over any particular horizontal station \(x\);
\item
  it is incompressible;
\item
  it is inviscid, and
\item
  any gravitational effects are negligible.
\end{itemize}

The volumetric flow rate in the nozzle is given as \(Q\) and the ambient
pressure is \(p_a\).

\begin{itemize}
\item
  Derive an expression for the gauge pressure at a horizontal station
  where the area is \(A\left(x\right)\).
\item
  Show by integrating the \(x-\)component of the pressure force on the
  nozzle's interior walls, that the net \(x-\)component of force on the
  nozzle due to the flow is independent of the specific nozzle contour
  and is given by
\end{itemize}

\[
\large
F = \rho Q^2 \frac{\left(A_{inlet} - A_2\right)^2}{2A_{inlet} A^2_2}
\]

where \(A_2\) is the area at the exit and \(A_{inlet}\) is the area at
the inlet.

Solution

While it is natural to apply Bernoulli's principle to such a problem,
let us clarify why. Consider all the assumptions that underscore
Bernoulli, i.e., the flow must be

\begin{itemize}
\tightlist
\item
  inviscid;
\item
  along a streamline (satisfied by the fact that we are told the flow is
  approximately uniform);
\item
  steady;
\item
  constant density;
\item
  no work / energy input or loss (e.g., there is no friction).
\end{itemize}

This enables us to write:

\[
\large
\require{color}
\underbrace{p\left(x\right) + \frac{1}{2}{\color[rgb]{0.918231,0.469102,0.038229}\rho} {\color[rgb]{0.059472,0.501943,0.998465}v}^2\left(x\right)}_{\textsf{station 1}} = \underbrace{{\color[rgb]{0.315209,0.728565,0.037706}p}_{a}  + \frac{1}{2}{\color[rgb]{0.918231,0.469102,0.038229}\rho}  {\color[rgb]{0.059472,0.501943,0.998465}v'}_{2}^2}_{\textsf{station 2}}
\]

\[
\large
\require{color}
{\color[rgb]{0.315209,0.728565,0.037706}p}\left(x\right) - {\color[rgb]{0.315209,0.728565,0.037706}p}_a = \frac{1}{2}{\color[rgb]{0.918231,0.469102,0.038229}\rho} \left( {\color[rgb]{0.059472,0.501943,0.998465}v}_2^2 - {\color[rgb]{0.059472,0.501943,0.998465}v}^2\left(x\right)\right)
\]

The gauge pressure \({\color[rgb]{0.315209,0.728565,0.037706}p}_{g}\) is
defined as the atmospheric pressure subtracted from a measured pressure,
i.e.,

\[
\large
\require{color}
{\color[rgb]{0.315209,0.728565,0.037706}p}_{g}  = {\color[rgb]{0.315209,0.728565,0.037706}p}\left( x \right) - {\color[rgb]{0.315209,0.728565,0.037706}p}_a
\]

\[
\large
\require{color}
= \frac{1}{2}\rho \left({\color[rgb]{0.059472,0.501943,0.998465}v}_2^2 - {\color[rgb]{0.059472,0.501943,0.998465}v}^2\left(x\right) \right) 
\]

Recognising that the velocity may be written as

\[
\large
\require{color}
{\color[rgb]{0.059472,0.501943,0.998465}v}\left(x\right) = \frac{Q}{A\left(x\right)}
\]

we have

\[
\large
\require{color}
{\color[rgb]{0.315209,0.728565,0.037706}p}_g = \frac{1}{2}{\color[rgb]{0.918231,0.469102,0.038229}\rho} \left({\color[rgb]{0.059472,0.501943,0.998465}v}_2^2 - \frac{Q^2}{A^2\left(x\right)} \right) = \frac{1}{2}{\color[rgb]{0.918231,0.469102,0.038229}\rho} Q^2 \left(\frac{1}{A_2^2} - \frac{1}{A^2\left(x\right)} \right) .
\]

The question asks us to integrate the \(x-\)component of the pressure
force, i.e.,

\[
\large
\require{color}
\int_{inlet}^{2^{\ast}} {\color[rgb]{0.315209,0.728565,0.037706}p}_{g}\left(x\right) dA = F_{pressure,x}
\]

We still need to work out the relevant length along the vertical
direction; see the Figure below.

\[
\large
\require{color}
F_{pressure,x}  = \int_{A_{inlet}}^{A_2} {\color[rgb]{0.315209,0.728565,0.037706}p}_{g} \; d\left[A_{inlet} - A \right]
\]

\[
\large
\require{color}
= -\int_{A_{inlet}}^{A_2} {\color[rgb]{0.315209,0.728565,0.037706}p}_{g} \; dA
\]

\[
\large
\require{color}
 = -\frac{{\color[rgb]{0.918231,0.469102,0.038229}\rho} Q^2}{2} \int_{A_{inlet}}^{A_2} \left(\frac{1}{A_2^2} - \frac{1}{A^2\left(x\right)} \right)dA 
\]

\[
\large
\require{color}
=-\frac{{\color[rgb]{0.918231,0.469102,0.038229}\rho} Q^2}{2} \left\{ \left[ \frac{A}{A_2^2}\right]_{A_{inlet}}^{A_2} - \left[- \frac{1}{A}\right]_{A_{inlet}}^{A_2}\right\}
\]

\[
\large
\require{color}
= -\frac{{\color[rgb]{0.918231,0.469102,0.038229}\rho} Q^2}{2} \left( \frac{A^2_{inlet} - 2A_{inlet}A_2 + A^2_{2}}{-A_{inlet}A_2^2} \right)
\]

\[
\large
\require{color}
= \frac{{\color[rgb]{0.918231,0.469102,0.038229}\rho} Q^2}{2} \frac{\left(A_{inlet} - A_2 \right)^2}{A_{inlet}A_2^2}
\]

Note that this pressure force always points in the right direction
regardless of whether \(A_2 > A_{inlet}\) or \(A_{inlet}> A_2\).

\hypertarget{problem-6}{%
\subsection{Problem 6}\label{problem-6}}

A circular nozzle is designed to produce a parallel jet of water (with
density
\({\color[rgb]{0.918231,0.469102,0.038229}\rho} ={\color[rgb]{0.918231,0.469102,0.038229} 1000 \; kg/m^3}\))
inclined at \(60^{\circ}\) to the entry direction, as shown in the
figure below. The entry and exit diameters are \(200 \; mm\) and
\(100 \; mm\) respectively. The water exists into the atmosphere with an
exit velocity of \({\color[rgb]{0.059472,0.501943,0.998465}8 \; m/s}\).
The flow through the nozzle can be assumed to be inviscid and gravity
can be neglected.

\begin{itemize}
\tightlist
\item
  Explain why the gauge pressure of the water is zero at the exit of the
  nozzle.
\item
  Calculate the velocity at the inlet to the nozzle and the mass flow
  rate of water.
\item
  Calculate the gauge pressure of the water at the entry to the nozzle.
\item
  Calculate the force on the nozzle due to the water flow. Express the
  answer in terms of the \(x\) and \(y\) coordinate system shown in the
  figure.
\end{itemize}

Solution

Solutions are captured below.

The question states that the nozzle is designed to produce a parallel
jet of water. Thus, at the exit we have parallel streamlines and
consequently no transverse pressure gradient.

\[
\large
\require{color}
A_1 = \frac{\pi }{4} \left(0.2\right)^2 = 0.031416 \; m^2, \; \; \; \; \; A_2 = \frac{\pi}{4} \left( 0.1 \right)^2 = 0.007854 \; m^2.
\]

The velocity at the entrance can be worked out via continuity. Since we
are given no information on the density, we assume the flow is
incompressible and therefore the volumetric flow rates are equal. This
yields

\[
\large
\require{color}
A_1 {\color[rgb]{0.059472,0.501943,0.998465}v}_1 = A_2 {\color[rgb]{0.059472,0.501943,0.998465}v}_2  \; \; \; \Rightarrow \; \; \; {\color[rgb]{0.059472,0.501943,0.998465}v}_1 = \frac{A_2 {\color[rgb]{0.059472,0.501943,0.998465}v}_2}{A_1} = \frac{0.1^2 \times {\color[rgb]{0.059472,0.501943,0.998465}8}}{0.2^2} = {\color[rgb]{0.059472,0.501943,0.998465}2 \; m/s}.
\]

To work out the massflow rate, we use continuity

\[
\large
\require{color}
\dot{m} = {\color[rgb]{0.918231,0.469102,0.038229}\rho} A_2 {\color[rgb]{0.059472,0.501943,0.998465}v}_2 = {\color[rgb]{0.918231,0.469102,0.038229}1000} {\color[rgb]{0.918231,0.469102,0.038229} \frac{kg}{m^3}} \times 0.007854 m^2 \times {\color[rgb]{0.059472,0.501943,0.998465}8} {\color[rgb]{0.059472,0.501943,0.998465} \frac{m}{s}} = 62.83 \frac{kg}{s}
\]

As this problem makes no statement on pressure losses within the
circular nozzle, we can write

\[
\large
\require{color}
{\color[rgb]{0.315209,0.728565,0.037706}p}_1 + \frac{1}{2}{\color[rgb]{0.918231,0.469102,0.038229}\rho} {\color[rgb]{0.059472,0.501943,0.998465}v}_1^2 = {\color[rgb]{0.315209,0.728565,0.037706}p}_2 + \frac{1}{2}{\color[rgb]{0.918231,0.469102,0.038229}\rho} {\color[rgb]{0.059472,0.501943,0.998465}v}_2^2
\]

\[
\large
\require{color}
\Rightarrow {\color[rgb]{0.315209,0.728565,0.037706}p}_1 = 0 + \frac{1}{2} {\color[rgb]{0.918231,0.469102,0.038229}\rho} \left( {\color[rgb]{0.059472,0.501943,0.998465}v}_2^2 - {\color[rgb]{0.059472,0.501943,0.998465}v}_1^2\right) 
\]

\[
\large
\require{color}
{\color[rgb]{0.315209,0.728565,0.037706}p}_1 = \frac{1}{2} \times 1000 \times \left(8^2 - 2^2\right) = 30 \; kPa 
\]

To calculate the forces, consider the diagram below.

Along the horizontal direction, we have the following steady flow
momentum equation

\[
\large
\require{color}
{\color[rgb]{0.315209,0.728565,0.037706}p}_1 A_1 - F_{x} - {\color[rgb]{0.315209,0.728565,0.037706}p}_2 A_2 cos\left(60^{\circ} \right) = \dot{m} {\color[rgb]{0.059472,0.501943,0.998465}v}_2 cos\left(60^{\circ}\right) - \dot{m} v_1 
\]

where we have used the gauge pressure. Along the vertical direction we
have

\[
\large
\require{color}
0 - F_y - {\color[rgb]{0.315209,0.728565,0.037706}p}_2 A_2 sin\left(60^{\circ}\right) = \dot{m}{\color[rgb]{0.059472,0.501943,0.998465}v}_2 sin\left( 60^{\circ}\right) - 0.
\]

Simplifying the above equations individually we have

\[
\large
\require{color}
F_{x}  = \left( {\color[rgb]{0.315209,0.728565,0.037706}p}_1 A_1 - {\color[rgb]{0.315209,0.728565,0.037706}p}_2 A_2 cos\left( 60^{\circ} \right) \right) + \dot{m}\left( {\color[rgb]{0.059472,0.501943,0.998465}v}_1 - {\color[rgb]{0.059472,0.501943,0.998465}v}_2 cos\left(60^{\circ} \right) \right)
\]

\[
\large
\require{color}
\left( 30000 \times 0.031416  - 0 \right) + 62.83 \left(2 - 8 \times 0.5 \right) 
\]

\[
\large
\require{color}
942.48 - 125.66 =  816.82 N
\]

and along the vertical direction

\[
\large
\require{color}
F_y = \left(-{\color[rgb]{0.315209,0.728565,0.037706}p}_2 A_2 sin\left( 60^{\circ}\right) \right) - \dot{m}{\color[rgb]{0.059472,0.501943,0.998465}v}_2 sin\left( 60^{\circ} \right) 
\]

\[
\large
\require{color}
0 - 62.83 \times 8 \times \sqrt{3}/2  = -435.80N
\]

\hypertarget{problem-7}{%
\subsection{Problem 7}\label{problem-7}}

The velocity
\(\require{color}{\color[rgb]{0.059472,0.501943,0.998465}v}\) at a
position \({\color[rgb]{0.986048,0.008333,0.501924}x}\) along the
center-line of the convergent entry nozzle of a wind tunnel operating at
steady state is given by

\[
\large
\require{color}
\mathbf{{\color[rgb]{0.059472,0.501943,0.998465}v}}  = \left({\color[rgb]{0.059472,0.501943,0.998465}v}_{{\color[rgb]{0.986048,0.008333,0.501924}x}}, {\color[rgb]{0.059472,0.501943,0.998465}v}_{{\color[rgb]{0.131302,0.999697,0.023594}y}}, {\color[rgb]{0.059472,0.501943,0.998465}v}_{{\color[rgb]{0.501963,0.000037,0.250983}z}} \right) \; \; \; 
\textrm{where} \; \;  \; \; {\color[rgb]{0.059472,0.501943,0.998465}v}_{{\color[rgb]{0.986048,0.008333,0.501924}x}} = \alpha\left(0.25 + 0.75 \frac{x}{L} \right), \; {\color[rgb]{0.059472,0.501943,0.998465}v}_{{\color[rgb]{0.131302,0.999697,0.023594}y}}= 0, \; \; \textrm{and} \; \; {\color[rgb]{0.059472,0.501943,0.998465}v}_{{\color[rgb]{0.501963,0.000037,0.250983}z}}=0.
\]

The nozzle has length \(\mathsf{L}\) and the velocity in the working
section is \(\alpha\).

\begin{enumerate}
\def\labelenumi{\arabic{enumi}.}
\item
  If \(L = 1.2\) m and \(\alpha=10\) m/s, what is the center-line
  velocity at \(x=0.4\) m?
\item
  What is the rate of change of the velocity field,
  \(\partial \mathbf{{\color[rgb]{0.059472,0.501943,0.998465}v}}/\partial t\)
  at this point? Include the units please.
\item
  Write down the definition of the material (also known as substantial)
  derivative, \(D / Dt\), and simplify it for the special case where
  \({\color[rgb]{0.059472,0.501943,0.998465}v}_{{\color[rgb]{0.131302,0.999697,0.023594}y}}= 0\)
  and
  \({\color[rgb]{0.059472,0.501943,0.998465}v}_{{\color[rgb]{0.501963,0.000037,0.250983}z}}= 0\)
  in Cartesian coordinates. Use this to find the acceleration of a fluid
  blob as it passes this point. Include the units.
\item
  Why are your answers to the above two questions different?
\end{enumerate}

Solution

Please find the solutions below:

\begin{enumerate}
\def\labelenumi{\arabic{enumi}.}
\item
  \({\color[rgb]{0.059472,0.501943,0.998465}v}_{{\color[rgb]{0.986048,0.008333,0.501924}x}} = 10 \times (0.25 + 0.75\frac{0.4}{1.2}) = 5 \; \textrm{m}/\textrm{s}\).
  Note that on the center-line
  \({\color[rgb]{0.059472,0.501943,0.998465}v}_{{\color[rgb]{0.131302,0.999697,0.023594}y}}\)
  and
  \({\color[rgb]{0.059472,0.501943,0.998465}v}_{{\color[rgb]{0.501963,0.000037,0.250983}z}}\)
  are both zero.
\item
  The flow is steady so
  \(\partial \mathbf{{\color[rgb]{0.059472,0.501943,0.998465}v}} / \partial t\)
  is zero at all points in the flow.
\item
  The material derivative is defined as:
\end{enumerate}

\[
\large
\require{color}
\frac{D}{Dt} = \left( \frac{\partial }{\partial t}  + \mathbf{{\color[rgb]{0.059472,0.501943,0.998465}v}} \cdot \nabla \right)
\]

In the special case where
\({\color[rgb]{0.059472,0.501943,0.998465}v}_{{\color[rgb]{0.986048,0.008333,0.501924}x}}\)
and
\({\color[rgb]{0.059472,0.501943,0.998465}v}_{{\color[rgb]{0.131302,0.999697,0.023594}y}}\)
are zero, this reduces to:

\[
\large
\require{color}
\frac{D}{Dt} = \left( \frac{\partial}{\partial t} + {\color[rgb]{0.059472,0.501943,0.998465}v}_{{\color[rgb]{0.986048,0.008333,0.501924}x}} \frac{\partial}{\partial {\color[rgb]{0.986048,0.008333,0.501924}x}} \right)
\]

The acceleration of a fluid blob as it passes this point is given by
\(D \mathbf{{\color[rgb]{0.059472,0.501943,0.998465}v}} / Dt\). We know
that
\({\color[rgb]{0.059472,0.501943,0.998465}v}_{\color[rgb]{0.131302,0.999697,0.023594}y}\)
and
\({\color[rgb]{0.059472,0.501943,0.998465}v}_{\color[rgb]{0.501963,0.000037,0.250983}z}\)
are both zero, so this reduces to
\(D{\color[rgb]{0.059472,0.501943,0.998465}v}_{{\color[rgb]{0.986048,0.008333,0.501924}x}} / Dt\):

\[
\large
\require{color}
\frac{D {\color[rgb]{0.059472,0.501943,0.998465}v}_{{\color[rgb]{0.986048,0.008333,0.501924}x}}}{Dt} = 0 + {\color[rgb]{0.059472,0.501943,0.998465}v}_{{\color[rgb]{0.986048,0.008333,0.501924}x}} \frac{0.75 \alpha}{L} = 5 \times \frac{0.75 \times 10}{1.2} = 31.25 \frac{m}{s^2}. 
\]

\begin{enumerate}
\def\labelenumi{\arabic{enumi}.}
\setcounter{enumi}{3}
\tightlist
\item
  Although the flow is steady at all points in space, fluid blobs do
  accelerate as the move through the flow field. This is the main point
  of the question.
\end{enumerate}



\end{document}
